\subsection {Session 5, Exercise 6}

\lineparagraph {Exercise}

Let $L_r$ be an arbitrary regular language and let $L_c$ be an arbitrary CF language.
\begin{enumerate}
    \item Show an example when $L_r \cap L_c$ is not regular.
    \item Prove that $L_r \cap L_c$ is always context-free.
    \item Show an example when $L_1$ and $L_2$ are both context-free but $L_1 \cap L_2$ is not.
\end{enumerate}

\lineparagraph {Solution}

a)

For example if we take $L_r = \Sigma^*$, which is regular. Then, we take a known non-regular, but CF language, $L_c = \{a^nb^n | n \geq{} 0\}$. Ther instersection is $L_c$ itself, which is not regular, but CF.

b) If $L_r$ is regular, there exists a DFA that accepts it, let's call this $M_r$. Then, since $L_c$ is CF, there exists a PDA that accepts it, let's call this $M_c$. To show that $L_r \cap L_c$ is CF we construct a PDA from $M_r$ and $M_c$ that accepts it.

The main idea is to take all the possible pairs of states, where one state comes from $M_r$ and the other from $M_c$. For a given input character, we define the transition function, so that ''it keeps track of what's happening in both $M_r$ and $M_c$ at the same time'', for each statepair $(q_r, q_c)$, by checking what $M_r$ would do for the given input character in state $q_r$ and what would $q_c$ do, and moving to that statepair (or, since the PDA can have a set of possible states it moves into, the set of all statepairs).

We keep track of what's happening in $M_c$'s stack in the stack we have (this is why we cannot do this for two CF-languages, we would need to keep track of two stacks).

The starting statepair is going to be the statepair which contains the starting states from their respective machines.

The accepting statepairs will be the ones for which both states accept in their respective machines, since we need $L_r \cap L_c$.

The PDA constructed in this way will accept $L_r \cap L_c$, which means that the language is CF.

c)

These languages are both CF:

$L_1 = \{a^nb^nc^k | n,k \geq{} 0\}$ (Number of $a$'s and $b$'s is equal.)

$L_2 = \{a^ib^jc^j | i,j \geq{} 0\}$ (Number of $b$'s and $c$'s is equal.)

(See \ref{4_3_d} for a CF-grammar for $L_1$, and based on that $L_2$ can be constructed in a similar manner.)

$L_1 \cap L_2 = \{a^nb^nc^n\}$  (Number of $a$'s and $b$'s and $c$'s is equal.)

Which is known to be non-CF. (The idea behind this is that we would need to use the stack to keep track of the number of $a$'s, but we would throw them out when comparing them with the number of $b$'s and there will be nothing left to compare to when the $c$'s come. The formal proof is more complicated and outside of the scope of this class.)