\subsection{Session 3, Exercise 11}

\lineparagraph{Exercise}

Determine the languages generated by the following grammars.

a.)
\begin{align*}
T &\rightarrow TT | aTb | bTa | a | \varepsilon
\end{align*}

b.)
\begin{align*}
R &\rightarrow TaT\\
T &\rightarrow TT | aTb | bTa | a | \varepsilon
\end{align*}

\lineparagraph{Solution}

a.)
\begin{align*}
T &\rightarrow TT | aTb | bTa | a | \varepsilon
\end{align*}

We can quickly see, that for any word generated by this grammar, the number of $a$'s can not be less than the number of $b$'s in it.

Intuitively we have a gut feeling, that due to the first rule, the order in which these characters appear might be completely arbitrary and actually all words like this can be generated.

We will show that this is true:

\textbf{Proof}

Statement: Any word for which the number of $a$'s is not less than the number of $b$'s can be generated using the production rules above.

We will be using mathematical induction for the length of the generated strings.

For the base case, of length either $1$ or $0$, the word can either be $\varepsilon$ (the empty string) or $a$, both of which can be generated (in one step, using either the fourth or the fifth rule).

Inductive step: We will prove that if the statement is true for any word of length less than $n$, then it is also true for any word of lengfth $n$.

Let's take a word of length $n$: $w = x_1x_2,\dots,x_n$.

Let's take the smallest $i$, for which in the word $w_{1..i}$ there is exactly as many $a$'s as $b$'s.

If there is no such $i$, but we are in the language, the only possible way is that all prefixes contain strictly more $a$'s, then $b$'s, specifically for $i=1$ as well, which means that $w_1 = a$. We can generate this letter by using the first and the fourth production rules as such $T \rightarrow TT \rightarrow aT$, where $T$ would have to generate the word $w_{2..n}$, which is of length $n-1$, which can be generated via $T$ according to the induction hypothesis.

If there is such $i$, then $x_i\neq{}x_1$, since $i$ is the first time the number of $a$'s is equal to the number of $b$'s.

Then, we can use the first, then either the second or the third production rule to generate the characters $x_1$ and $x_i$, such as $T \rightarrow TT \rightarrow x_1Tx_iT$. Where the remainder two parts of the word $w_{2..i-1}$ and $w_{i+1..n}$ are also 
in the language and can be generated using $T$, since their lenght is less than $n$, according to the induction hypothesis.

b.)
\begin{align*}
R &\rightarrow TaT\\
T &\rightarrow TT | aTb | bTa | a | \varepsilon
\end{align*}

$T$ is the same as in a.), and due to $R$, now the number of $a$'s must be strictly greater than the number of $b$'s.