\subsection {Session 9, Exercise 4}

\lineparagraph {Exercise}

There are $n$ people working at an organization, where there are $c$ committees. We want to schedule committee
meetings. Two of the committees may have a meeting on the same day if they have no common members.
Let positive integer $k$ be given and the lists of members of each committee. We want to determine whether the
meetings of all committees can be scheduled in a period of $k$ days. Either give a polynomial time algorithm
that solves this problem or prove that it is NP-complete.

\lineparagraph {Solution}

\begin{itemize}
    \item It is NP-complete. Scheduling problems are usually related to coloring. In this case, we will solve the $COLOR$ problem using a solver for $L$.
    \item The Witness Theorem can be used to prove that it is in NP, give a witness, a vertifyer algorithm and prove that the witness is polynomial in size and he verifyer runs in polynomial time.
    \item The input of $COLOR$ is graph $G$ and the $m$ number of colors we want to use to color it. We will translate each vertex of the graph $G$ into a committee, so $v_i$ vertex to $c_i$ comittee.
    \item Then, where there is an edge running between two vertices, that will correspond to a person, who is present in both comittees. We assign $p_i$ person to $e_i$, edge, add $p_i$ to the comitees corresponding to both ends of the edge.
    \item Then, we will say that we want to schedule meetings in $k=m$ days, each day corresponding to a single color of the original graph.
    \item If two committees can be scheduled to meet at the same day, that means they don't have common members, so there is no edge running between their vertices.
    \item So a proper scheduling onto $k$ days corresponds to a proper coloring of the graph using $k$ colors.
    \item If no proper scheduling is possible onto $k$ days, then no $k$ coloring is possible, either. If a coloring was possible, that would correspond to a scheduling as well.
    \item It can be shown that this transformation is also polynomial.
\end{itemize}

