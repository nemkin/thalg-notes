\subsection {Session 9, Exercise 3}

\lineparagraph {Exercise}

 Is the following problem in P or is it NP-complete? We are given integer numbers $a1, \dots{}, a_n$ and question
is whether this set of numbers can be partitioned into three parts so that sum of the numbers in each part
is the same.

\lineparagraph {Solution}

\begin{itemize}
    \item This smells like an NP-complete problem, because it looks similar to $PARTITION$.
    \item First prove that it is in NP using the Witness Theorem:
    \item The witness will be a description of such a partition, which is given by a list of $n$ numbers between $1$ and $3$ the $i$th representing which partition the $a_i$ belongs to. This is polynomial, $O(n)$ size.
    \item Witness checking will be summing the partitions separately and comparing that they indeed sum to the same number and also that no index number left out (by giving a number other than $1,2,3$ on the input or giving less than $n$ numbers on the input). This can be done in polynomial time.
    \item Now prove that it is NP-hard by Karp-reducing the well-known $PARTITION$ $NP$-complete problem onto it.
    \item We can take the sum of all $a_i$'s from the input of $PARTITION$ and divide by $2$. Add this $c$ value to the list of $a_i$'s and give this list to the solver for $L$.
    \item The solver must put $c$ into a a separate group, since if it tried to put additional numbers next to it, the group would become too big, and the remaining numbers won't be able to balance it out.
    \item So if there is a $3-way$ partition of these numbers to an equal sum, it can only be done by $c$ being a separate group, meaning that the other two groups will be a $2-way$ partition of the original input.
    \item The transformation here is polynomial as well, we sum $n$ nubmers and do some additional arithmetic, done in $O(n)$.
\end{itemize}
