\subsection{Exam: 2022. 06. 13., Exercise 5}

\lineparagraph{Exercise}

We are given the array $A[1:n]$, of length $n$, which contains \textbf{unique} integers. We want to decide, whether or not we can select $100$ different numbers from the array, such that the difference of any pairs will be at most $2022$. Give a $O(n \log n)$ runtime algorithm to solve this problem!

\lineparagraph{Solution}

\begin{itemize}
    \item The first thought here is: $O(n \log n)$ runtime allowed $\rightarrow$ we can sort! And that will help!
    \item Using e.g. merge sort, which runs in $O(n \log n)$ time we sort the array.
    \item It will help, because after sorting, values that are closer to each other in difference, will be places physically closer to each other.
    \item This means that after sorting, to find $100$ numbers for which all pairs of numbers differ by at most $2022$, we can look at a consecutive subarray of length exactly $100$ and check if the first and the last element of this subarray meet this requirement.
    \item If they do, then all pairs inside the subarray also do, their difference can be equal or smaller.
    \item If there is no solution like this, then there is no possible solution either way. If a solution of size $100$ exists, then a solution which is $100$ consecutive numbers in the sorted array must also exist. Any gaps in a non-consecutive solution contain numbers that also meet the criteria.
    \item So in the sorted array we go from $k=1$ to $k=n-99$ and check if $|A_{\text{sorted}}[k] - A_{\text{sorted}}[k+99]| \leq{} 2022$. If we find such a $k$, then numbers $\{A_{\text{sorted}}[k], \cdots{}, A_{\text{sorted}}[k+99]\}$ are the solution.
    \item If we find no such $k$, then no solution exists.
\end{itemize}