\subsection {Session 11, Exercise 1}

\lineparagraph {Exercise}

Consider the version of problem $SUBSETSUM$ where $a_1, a_2, \dots{}, a_n$ and $b$ are integers, such that $0<a_i<n^2$ for all $1<leq{}i\leq{}n$. The question is that does there exist a subset $I\subseteq{}\{1,\dots{}n\}$, such that $\sum\limits_{i\in{}I}a_i = b$. Prove that this version of the problem is in $P$.

\lineparagraph {Solution}

\begin{itemize}
    \item So if all weights are limited to $n^2$ (where $n$ is the number of weights), then the problem should be easier than the original. Why is that? Why is the $n^2$ limit helpful to us?
    \item First of all, if a single number is limited to $n^2$, then the total sum is limited to at most $n^3$.
    \item If you remember from the $KNAPSACK$ solver (which is very similar to $BINPACKING$), it was a DP algorithm that created a table of size $n$ (corresponding to the number of items) times $b$ (corresponding to the weight limit there). But $b$ is was not polynomial relative to the input size, which is why this algorithm was exponential (the input size of the number $b$ is $O(log_2(b))$).
    \item This problem is now solved, by $b$ being limited by $n^3$, so the DP table size is limited to $O(n\cdot{}n^3)$. (If the input $b$ is larger than $n^3$, we return ''NOT POSSIBLE'', without making a table.)
    \item So this tells us that we should create a solver, similar to Knapsack, that in this case will run in polynomial time.
    \item The \textbf{key statement} for this DP solver is the following: \textbf{To reach the sum $b$, we either use the number $a_n$ or not use it.}
    \item Since we are looking for a recursive formula, it makes sense to reduce to the smaller size problem of $\{a_1,\dots{}, a_{n-1}\}$ and either $b$ or $b-a_n$, depending on whether we wanted to use $a_n$ or not. So usually the key statement will look at and remove the last number from the input, not the first.
    \item If we use $a_n$, this means that we need to get the sum of $b-a_n$ using the values $\{a_1,\dots{}, a_{n-1}\}$, while if we don't use $a_n$, we need to get the sum $b$ using the values $\{a_1,\dots{},a_{n-1}\}$.
    \item We need to keep track of both the current sum and the index $i$ of the last $a_i$ number, so we define $T[i,c]$ to mean: can we achieve the sum $c$ using values $\{a_1,\dots{}a_i\}$? ($\rightarrow$ \textbf{Step 1}: Define the DP table. What does $T[i,c]$ mean?)
    \item The recursive formula is the following:
\end{itemize}

\begin{align*}
    T[i,c] = T[i-1, c-a_i] \lor{} T[i-1, c]
\end{align*}

\begin{itemize}
\item If either sum is possible, then $T[i,c]$ is also possible. $\lor{}$ means the boolean or operator.  ($\rightarrow$ \textbf{Step 2}: Generic recursive formula.)
\item The base cases is where we index out of the table $T$. This can happen if $i=1$, and also there are a lot of negative indexes in the second dimension.
\item Let's just say, that if in $T[i,j]$, the second index is negative then we know that whatever sum we are trying to reach is not possible like that, so it should be a $False$ value. So define $T[i,j] = False$ if $j<0$.
\item Then for $T[1.j]$ we are trying to get a sum of exactly $j$, using only $a_1$. This is only possible if $j=a_1$. So $T[1,a_1] = True$, and false for any other $j$. ($\rightarrow$ \textbf{Step 3}: Base cases.)
\item The order in which we must fill out $T$: We want to go row-by-row starting from smaller to larger indexes, since the recursive formula is using the previous row's values. Then inside a row, we go from smaller to larger index columns. There is no need to fill $T$ further than the $b$th column, since we don't need those values for the sum of $b$. ($\rightarrow$ \textbf{Step 4}: Filling order.)
\item The solution is in $T[n,b]$, meaning if the sum of $b$ is possible using any subset of all of the values.
\item We fill this table in $O(n\cdot{}n^3)$ time, since the size of the table is $n\cdot{}n^3$ and we do $O(1)$ operations for each cell. ($\rightarrow$ \textbf{Step 6}: Runtime complexity.)
\item This is a polynomial algorithm for this specific case of the $BINPACKING$ problem, so this proves that this special case is in $P$. (The general case is still $NP$-complete though!)
\end{itemize}