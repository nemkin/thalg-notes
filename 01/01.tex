\subsection{Session 1, Exercise 1}

\lineparagraph{Exercise}

There are two algorithms for the same problem, $A$ and $B$. Let the functions describing their worst case running times be $f_A$ and $f_B$. It is known that $f_A(n) = O(f_B(n))$. Does it follow that...

\begin{enumerate}[a.)]
\item $A$ is faster on every input than $B$?
\item except for finitely many inputs $A$ is faster than $B$?
\item for large enough inputs $A$ is faster than $B$?
\end{enumerate}

\lineparagraph{Solution}

The answer is \textbf{no}, in all three cases.

A few good counterexamples:
\begin{itemize}
    \item $f_A(n) = 2n$ and $f_B(n)=n$. Even though $f_A(n)\in{}O(f_B(n))$, actually $f_A(n) > f_B(n)$, so $A$ is worst case slower than $B$.
    \item Any functions where $f_a(n) = f_b(n) = f(n)$, since $f(n)\in{}O(f(n))$.
    \item For just exercise $a.)$, $f_A(n) = n$ and $f_B(n)=\frac{n^2}{4}$, since $n\in{}O(\frac{n^2}{4})$, however for $n=1,2,3$ $f_A(n)>f_B(n)$.
\end{itemize}


