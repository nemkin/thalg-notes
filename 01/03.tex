\subsection{Session 1, Exercise 3}

\lineparagraph{Exercise}

For which integers $a,b > 1$ do the following hold true?

\begin{enumerate}[a.)]
\item $n^a = O(n^b)$
\item $2^{an} = O(2^{bn})$
\item $\log_a(n) = O(\log_b(n))$
\end{enumerate}

\lineparagraph{Solution}

 \textbf{a)}

$n^a = O(n^b)$ holds true if there exists $c>0$ and $n_0\in{}\mathds{Z}^+$, such that $n^a \leq{} cn^b$, for $\forall{}n\geq{}n_0$.

Then by moving $n^b$ to the left, we get:

$\frac{n^a}{n^b} \leq{} c$, for $\forall{}n\geq{}n_0$.

Or

$n^{a-b}\leq{} c$, for $\forall{}n\geq{}n_0$.

Now, depending on the exponent $n^{x}$ can behave very differently as $n$ grows:

\begin{itemize}
    \item For $x > 0$ $n^{x}$ grows arbitrarily large, and cannot be upper bound with a constant.
    \item For $x=0$, $n^0 = 1$ is a constant and can be upper bound.
    \item For $x<0$, $n^{x} = n^{-|x|} = \frac{1}{n^{|x|}}$, where while $n^{|x|}$ grows arbitrarily large $\frac{1}{n^{|x|}} \rightarrow \frac{1}{\infty} = 0$. So for a large enough $n$, we can upper bound this with a constant. (For example $x^{-2} \leq{} \frac{1}{4}$ for $\forall{}n\geq{}2$.)
\end{itemize}

To sum up, if $x = a-b \leq{} 0$, or simply when $a\leq{}b$, then $n^a = O(n^b)$ holds true, otherwise not.

\textbf{b)}

$2^{an} = O(2^{bn})$ holds true if there exists $c>0$ and $n_0\in{}\mathds{Z}^+$, such that $2^{an} \leq{} c2^{bn}$, for $\forall{}n\geq{}n_0$.

Again, let's move $2^{bn}$ to the left:

$\frac{2^{an}}{2^{bn}} \leq{} c$, for $\forall{}n\geq{}n_0$.

Or

$2^{(a-b)n} \leq{} c$, for $\forall{}n\geq{}n_0$.

Again, depending on the exponent's multiplier $2^{xn}$ can behave very differently as $n$ grows:

\begin{itemize}
    \item If $x > 0$, then $2^{xn}$ grows to infinity, or can be arbitrarily large.
    \item If $x = 0$, then $2^{0n} = 1$, which can be upper bound with a constant.
    \item If $x < 0$, then $2^{xn} = 2^{-|x|n} = \frac{1}{2^{|x|n}}$, where while $2^{|x|n}$ grows arbitrarily large $\frac{1}{2^{|x|n}} \rightarrow \frac{1}{\infty} = 0$. So for a large enough $n$, we can upper bound this with a constant. (For example $2^{-n} \leq{} 1$ for $\forall{}n\geq{}1$.)
\end{itemize}

\textbf{c)}

$\log_a(n) = O(\log_b(n))$ holds true if there exists $c>0$ and $n_0\in{}\mathds{Z}^+$, such that $\log_a(n) \leq{} c\log_b(n)$, for $\forall{}n\geq{}n_0$.

Looking at this we already know that since $a$ is the base of a logarithm, it must be positive, $a>0$ and $a\neq{}1$ (the base of a logarithm cannot be $1$). Similarly $b>0$ and $b\neq{}1$.

Let's move $log_b(n)$ to the left:

$\frac{\log_a(n)}{\log_b(n)} \leq{} c$, for $\forall{}n\geq{}n_0$.

Here, we can use the change of base rule for logarithms:

$\log_a(n) = \frac{\log_b(n)}{\log_b(a)}$.

Let's substitute this into the previous formula:

$\frac{\frac{\log_b(n)}{\log_b(a)}}{\log_b(n)} \leq{} c$, for $\forall{}n\geq{}n_0$.

Now, since both the numerator and denominator contains $\log_b(n)$, we can reduce with it, to arrive at:

$\frac{\frac{1}{\log_b(a)}}{1} \leq{} c$, for $\forall{}n\geq{}n_0$.

And just leave the outer denominator:

$\frac{1}{\log_b(a)} \leq{} c$, for $\forall{}n\geq{}n_0$.

Now if the original statements are true ($a>0$, $a\neq{}1$, $b>0$, $b\neq{}1$), that means that the number $\frac{1}{\log_b(a)}$ exists and is a constant (since $a$ and $b$ are also constants), which can be upper estimated with a positive $c$, any $c$ works.
