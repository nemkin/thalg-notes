\subsection{Session 1, Exercise 4}

\lineparagraph{Exercise}

Put the following functions in order of non-decreasing order of magnitude, i.e., if $f_i$ is immediately followed by $f_j$, then $f_i(n) = O(f_j(n))$ holds.

\begin{enumerate}[a.)]
\item $f_1(n) = 8n^3$
\item $f_2(n) = 5\sqrt{n} + 1000n$
\item $f_3(n) = 2^{(\log_{2}n)^2}$
\item $f_4(n) = 1514n^{2}\log_{2}n$
\end{enumerate}

\lineparagraph{Solution}

Gut feeling: Firstly, we just look at the functions and what we see if we squint is:

\begin{enumerate}[a.)]
\item $f_1(n) \approx n^3$
\item $f_2(n) \approx n$
\item $f_3(n)$ well this one is complicated.
\item $f_4(n) \approx n^{2}\log_{2}n$
\end{enumerate}

Let's work on $f_3(n)$ a bit:

$f_3(n) = 2^{(\log_{2}n)^2} = 2^{(\log_{2}n)(\log_{2}n)} = (2^{(\log_{2}n)})^{(\log_{2}n)} = n^{\log_{2}n}$.

Now if we put this back in:

\begin{enumerate}[a.)]
\item $f_1(n) \approx n^3$
\item $f_2(n) \approx n$
\item $f_3(n) = n^{\log_{2}n}$
\item $f_4(n) \approx n^{2}\log_{2}n$
\end{enumerate}

The order should be something along the lines of $n << n^{2}\log_{2}n << n^3 << n^{\log_{2}n}$, since multiplying $n^2$ by $log_2n$ results in asymptotically a bit bigger than $n^2$, but not enough to reach $n^3$ and $n^{log_2n}$ has a non-constant exponent, while $n^3$'s exponent is constant, so between the two, $n^{log_2n}$ is asymptotically bigger.

Plugging back the function's names, we need to prove that $f_2(n) << f_4(n) << f_1(n) << f_3(n)$. This can be done by proving these three things:

\begin{itemize}
    \item $f_2(n) \in O(f_4(n))$
    \item $f_4(n) \in O(f_1(n))$
    \item $f_1(n) \in O(f_3(n))$
\end{itemize}

And the transitivity of $O$ means that all relations hold true.

\textbf{Let's begin with $f_2(n) \in O(f_4(n))$:}

$5\sqrt{n} + 1000n \in O(1514n^{2}\log_{2}n)$:

We need to find $c$ and $n_0$ so that
$5\sqrt{n} + 1000n \leq{} c1514n^{2}\log_{2}n$ for $\forall{}n\geq{}n_0$.

Let's work both sides individually:

$5\sqrt{n} + 1000n \leq{} 1005n$, since $\sqrt{n} \leq{} n$ when $n\geq{}1$.

$c1514n^{2}\log_{2}n \geq{} c1514n^{2}$, since $\log_2{n} \geq{} 1$ when $n\geq{}2$.

If we put them together:

$5\sqrt{n} + 1000n \leq{} 1005n \leq{} c1514n^{2} \leq{} c1514n^{2}\log_{2}n$ when $n\geq{}2$.

Just the middle part:

$1005n \leq{} c1514n^{2}$ when $n\geq{}2$.

Divide by $n$ and move the constants around:

$\frac{1005}{c1514} \leq{} n$ when $n\geq{}2$.

We can just set $c=1$, which means that $\frac{1005}{1514} \leq{} n$ must hold true, and it does, because we already have $n\geq{}2$ as a requisite. So $n_0=2$ and $c=1$ works.

\textbf{Let's continue with $f_4(n) \in O(f_1(n))$:}

$1514n^{2}\log_{2}n \in{} O(n^3)$, since the righthand side is already pretty clean we can just work on the lefthand side.

$1514n^{2}\log_{2}n \leq{} 1514n^3$ since $\log_{2}n\leq{}n$, for any $n$. Then we can choose $c=1514$ and $n_0=1$.

\textbf{Finally let's show that $f_1(n) \in O(f_3(n))$:}

$n^3 \in{} O(n^{\log_{2}n})$

$n^3 \leq{} cn^{\log_{2}n}$.

Let's just set $c=1$, the functions are easier to handle and work on finding an appropriate $n_0$.

When $n\geq{}1$ for a given $n$ this holds true if $3 \leq{} \log_{2}n$ which is true when $2^3 \leq{} n$, or $8 \leq{} n$. So $c=1$ and $n_0 = 8$ works.