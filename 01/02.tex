\subsection{Session 1, Exercise 02}

\lineparagraph{Exercise}

Which of the following functions are $O(n^2)$ and which are $\Omega(n^2)$?

\begin{enumerate}[a.)]
\item $f_1(n) = 11n^2 + 100000$
\item $f_2(n) = 8n^2\log_2(n)$
\item $f_3(n) = 1.5n + 3\sqrt{n}$
\end{enumerate}

\lineparagraph{Solution}

Definitions:

\begin{itemize}
    \item $f(n)\in{}O(g(n))$ if there exists $c>0$ and $n_0\in{}\mathds{Z^+}$ so that $f(n) \leq{} cg(n)$ for $\forall{}n\geq{}n_0$.
    \item $f(n)\in{}\Omega(g(n))$ if there exists $c>0$ and $n_0\in{}\mathds{Z^+}$ so that $f(n)\text{ }\color{red}\mathbf{\geq{}}\color{black} cg(n)$ for $\forall{}n\geq{}n_0$.
\end{itemize}

To prove that something is $\in{}O(g(n))$ or $\in{}\Omega(g(n))$, give one $c$ and $n_0$ and show that the above requirement is holds true.

To prove that something is not $\in{}O(g(n))$ or $\in{}\Omega(g(n))$, prove that no possible $c$ and $n_0$ combination would make the above requirement hold true.

Fist look at the given function, and figure out whether you think the definition is true or not and then based on that choose the appropriate proof.

\textbf{a)}

$f(n) = 11n^2 + 100000$.

Gut feeling: The most significant component here is $n^2$, so we believe it is asymptotically similar to $n^2$, so both $f(n)\in{}O(n^2)$ and $f(n)\in{}\Omega(n^2)$.

First let's show that $f(n)\in{}O(n^2)$:

Start from the function and give \textbf{upper} bounds until we reach something in the form of $\text{const}*n^2$.

$f(n) = 11n^2 + 10^5 \leq{} 12n^2$ if $n^2 \geq{} 10^5$, or $n \geq{} 10^{5/2}$, where we can be generous and just say $n \geq{} 1000$, since it is not necessary to find the smallest possible $n_0$, only one sufficient.

We have arrived at: $f(n) \leq{} 12n^2$ if $n\geq{}1000$.

Then by choosing $n_0=1000$ and $c=12$, this turns into $f(n) \leq{} cn^2$ if $n\geq{}n_0$, which by definition means that $f(n) \in{} O(n^2)$.

Now let's show that $f(n)\in{}\Omega(n^2)$:

Start from the function again, however now we give \textbf{lower} bounds until we reach something in the form of $\text{const}*n^2$.

$f(n) = 11n^2 + 10^5 \geq{} 11n^2$, since $10^5\geq{}0$.

We have arrived at: $f(n) \geq{} 11n^2$. Now $c=11$, and there was no requirement for $n$, so we can choose $n_0=1$, and then this turns into $f(n) \geq{} cn^2$ if $n\geq{}n_0$, which by definition means that $f(n)\in{}\Omega(n^2)$.

Note:
\begin{itemize}
    \item Since we have shown that $f(n)\in{}O(n^2)$ and also $f(n)\in{}\Omega(n^2)$, the combination of the two is that $f(n)\in{}\Theta(n^2)$, so $11n^2+100000$ and $n^2$ are asymptotically similar.
\end{itemize}

\textbf{b)}

$f(n) = 8n^2\log_2(n)$

Gut feeling: This is going to be asymptotically bigger than $n^2$, since it is also multiplied with $\log_2(n)$. So we believe $f(n)\in{}\Omega(n^2)$, however $f(n)\notin{}O(n^2)$.

Let's start with showing that $8n^2\log_2(n)\notin{}O(n^2)$:

In this case we want to show that there is no possible choice of $c$ and $n_0$, which is done using indirect proof and arriving at a contradiction.

Let's indirectly state that $8n^2\log_2(n)\in{}O(n^2)$, which is equivalent to saying:

There exists some $c>0$ and $n_0\in{}\mathds{Z^+}$ constants, so that

$8n^2\log_2(n) \leq{} cn^2$ for $\forall n\geq{}n_0$.

Let's divide the left and the righthand side with $n^2$:

$8\log_2(n) \leq{} c$ for $\forall n\geq{}n_0$.

Let's move the $8$ to the right:

$\log_2(n) \leq{} \frac{c}{8}$ for $\forall n\geq{}n_0$.

And this is a contradiction, since $\frac{c}{8}$ is a constant, while $\log_2(n)$ can be arbitrarily large for a given $n\geq{}n_0$, and we cannot give an upper bound to something arbitrarily large with a constant.

This means that the indirect statement was wrong so that $8n^2\log_2(n)\notin{}O(n^2)$.

Now let's show that $f(n)\in{}\Omega(n^2)$:

Start from the function again and give \textbf{lower} bounds until we reach something in the form of $\text{const}*n^2$.

$f(n) = 8n^2\log_2(n) \geq{} 8n^2$ when $n\geq{}2$, since $\log_2(n) \geq{} 1$ when $n\geq{}2$.

We have arrived at $f(n) \geq{} 8n^2$ when $n\geq{}2$, so we can choose $c=8$ and $n_0=2$, which gives $f(n) \geq{} cn^2$ when $n\geq{}n_0$, which is the definition of $f(n)\in{}\Omega(n^2)$.

\textbf{c)}

$f(n) = 1.5n + 3\sqrt{n}$

Gut feeling: This is going to be asymptotically smaller than $n^2$, since the most significant component is only $n$. So we believe $f(n)\in{}O(n^2)$, however $f(n)\notin{}\Omega(n^2)$.

Let's show that $f(n)\in{}O(n^2)$:

Start from the function and give \textbf{upper} bounds until we reach something in the form of $\text{const}*n^2$. (Actually, in this case, an even stonger statement will be made.)

$f(n) = 1.5n + 3\sqrt{n} \leq{} 4.5n$, since $\sqrt{n} \leq{} n$ (for $n\geq{}1$, which is a given).

We have arrived at $f(n) \leq{} 4.5n$, which means that $c=4.5$ and since there is no lower bound on $n$, we can choose $n_0=1$, to turn it into $f(n) \leq{} cn$ for $\forall{}n\geq{}n_0$, which means by definition that $f(n)\in{}O(n)$. 

Now since we know that $O(n)\subset{}O(n^2)$, this also proves that $f(n)\in{}O(n^2)$.

Let's continue with showing that $1.5n + 3\sqrt{n}\notin{}\Omega(n^2)$:

In this case we want to show that there is no possible choice of $c$ and $n_0$, which is done using indirect proof and arriving at a contradiction.

Let's indirectly state that $1.5n + 3\sqrt{n}\in{}\Omega(n^2)$, which is equivalent to saying:

There exists some $c>0$ and $n_0\in{}\mathds{Z^+}$ constants, so that

$1.5n + 3\sqrt{n} \geq{} cn^2$ for $\forall n\geq{}n_0$.

How we handle this situation is by giving something that is larger than $1.5n + 3\sqrt{n}$ and show that even this larger thing still cannot upper bound $cn^2$, so the smaller, original function can not either:

Borrowing from the previous proof, we know that $4.5n \geq{} 1.5n + 3\sqrt{n}$:

$4.5n \geq{} 1.5n + 3\sqrt{n} \geq{} cn^2$ for $\forall n\geq{}n_0$.

Now let's just look at the two sides:

$4.5n \geq{} cn^2$ for $\forall n\geq{}n_0$.

Let's divide both sides by $n$:

$4.5 \geq{} cn$ for $\forall n\geq{}n_0$.

And move the constant to the left:

$\frac{4.5}{c} \geq{} n$ for $\forall n\geq{}n_0$.

And here is the contradiction: this can not be true, since $n$ is arbitrarily large, while it should be upper bound by a constant: $\frac{4.5}{c}$. This shows that $4.5n$ can not be an upper bound for $cn^2$, so an even smaller function, $1.5n + 3\sqrt{n}$ definitely can't be either for any given $c$ and $n_0$ pair.