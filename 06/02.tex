\subsection{Session 6, Exercise 2}

\lineparagraph{Exercise}

A 2-szalagos $M$ Turing-gép átmeneti függvényét a következő táblázat írja le, ahol * jelöli a szalagon az üres jelet és $q_{0}$ a kezdő állapotot:

\begin{tabular}{|c|c|c||cc|cc|c|}
\hline állapot & 1. szalag & 2. szalag & 1. szalag & 2. szalag & új állapot \\
\hline$q_{0}$ & 0 & $*$ & 0 & $\mathrm{H}$ & $\mathrm{X}$ & $\mathrm{J}$ & $q_{1}$ \\
& 1 & $*$ & 1 & $\mathrm{H}$ & $\mathrm{X}$ & $\mathrm{J}$ & $q_{1}$ \\
& $*$ & $*$ & $*$ & $\mathrm{H}$ & $*$ & $\mathrm{H}$ & $q_{5}$ \\
\hline$q_{1}$ & 0 & $*$ & 0 & $\mathrm{~J}$ & 0 & $\mathrm{~J}$ & $q_{1}$ \\
& 1 & $*$ & 1 & $\mathrm{~J}$ & 1 & $\mathrm{~J}$ & $q_{1}$ \\
& $*$ & $*$ & $*$ & $\mathrm{H}$ & $*$ & $\mathrm{~B}$ & $q_{2}$ \\
\hline$q_{2}$ & $*$ & 0 & $*$ & $\mathrm{H}$ & 0 & $\mathrm{~B}$ & $q_{2}$ \\
& $*$ & 1 & $*$ & $\mathrm{H}$ & 1 & $\mathrm{~B}$ & $q_{2}$ \\
& $*$ & $\mathrm{X}$ & $*$ & $\mathrm{~B}$ & $\mathrm{X}$ & $\mathrm{J}$ & $q_{3}$ \\
\hline$q_{3}$ & 0 & 0 & 0 & $\mathrm{H}$ & 0 & $\mathrm{~J}$ & $q_{4}$ \\
& 1 & 1 & 1 & $\mathrm{H}$ & 1 & $\mathrm{~J}$ & $q_{4}$ \\
\hline$q_{4}$ & 0 & 0 & 0 & $\mathrm{~B}$ & 0 & $\mathrm{H}$ & $q_{3}$ \\
& 0 & 1 & 0 & $\mathrm{~B}$ & 1 & $\mathrm{H}$ & $q_{3}$ \\
& 1 & 0 & 1 & $\mathrm{~B}$ & 0 & $\mathrm{H}$ & $q_{3}$ \\
& 1 & 1 & 1 & $\mathrm{~B}$ & 1 & $\mathrm{H}$ & $q_{3}$ \\
& 0 & $*$ & 0 & $\mathrm{H}$ & $*$ & $\mathrm{H}$ & $q_{5}$ \\
& 1 & $*$ & 1 & $\mathrm{H}$ & $*$ & $\mathrm{H}$ & $q_{5}$ \\
\hline
\end{tabular}

(a) Mi a 2. szalag tartalma, amikor a gép $q_{2}$ állapotba kerül?

(b) Mi az $L(M)$ nyelv, ha $q_{5}$ az egyetlen elfogadó állapot?

(c) Legfeljebb hány lépést tehet a gép egy $n$ hosszú bemeneten, mielőtt megáll?

\lineparagraph{Solution}

Nézzük, mi történik:

$q_{0}:$ Ha nem üres a bemenet, akkor egy $X$ kerül a 2. szalagra (ez jelzi majd a szalag elejét).

$q_{1}$ : Lemásolja az 1. szalagon talált karaktert a 2. szalagra. Akkor lép csak ki innen, ha a bemenet végére ért (ekkor, ha az elsö szalagon a $w$ szó van, akkor a 2. szalagon $X w$ ). Ilyenkor következik a $q_{2}$ állapot, ahova úgy lépünk át, hogy az 1. szalagon maradunk az első üres mezón, a 2. szalagon visszalépünk az utolsónak írt karakterre.

$q_{2}:$ Az első szalagon nem mozdulunk, mialatt visszamegyünk a 2 . szalag elejére. Végül úgy lépünk át $q_{3}$-ba, hogy az 1. szalagon egyet visszalépünk (az utolsó nem üres mezóre), a 2. szalagon egyet elóre lépünk (az elsö nem $X$ karakterre). $q_{3}$ : Ha ugyanazt látjuk mindkét szalagon, akkor az elsőn nem mozdulunk, a másodikon egyet előre lépünk és átkerülünk $q_{4}$-be. (Elsớ alkalommal tehát az 1. szalag utolsó és a 2. szalag $X$ utáni elsö karakterét, azaz a $w$ első és utolsó karakterét hasonlítjuk össze. Ha nem egyeznek, akkor ebben a nem elfogadó állapotban leállunk.)

$q_{4}$ : Ha nem értük el a 2. szalag végét, akkor semmit nem változtatva az 1. szalagon visszafelé lépünk egyet, a 2. szalagon helyben maradunk és a $q_{3}$-ban folytatjuk. Azaz a $q_{3}$-beli hasonlítás és a $q_{4}$-beli lépés felváltva addig történik, amíg el nem érünk a 2. szalag végére. Ekkor átlépünk a q elfogadó állapotba és a számítás sikeresen véget ért.

Az üres szón érdemi munka nélkül egyból átjutunk az elfogadó állapotba.

(a) Ha $w$ a bemeneti szó, akkor $X w$.

(b) A palindromok nyelve.

(c) Egy $n$ hosszú palindromon a lépések száma: $1+(n+1)+(n+1)+(2 n-1)+1=4 n+3$. Ha a szó nem palindrom, akkor elóbb elakadhat (de a másolás és a 2. szalag elejére visszalépegetés akkor is megtörténik, $2 n+3$ lépés akkor is lesz, amennyiben $n>0)$.