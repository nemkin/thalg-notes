\subsection {Session 8, Exercise 12}
\label{8_12}

\lineparagraph {Exercise}

Let us assume that we have a procedure that decides about any n-vertex graph whether it has a Hamiltonian
cycle. How can this procedure be used to find a Hamiltonian cycle of a given graph $G$ in polynomial time?

\lineparagraph {Solution}

\begin{itemize}
    \item Similar to \nameref{8_11}.
    \item We iterate over the edges of the graph. For each edge, we remove it from the graph and ask the solver if the graph still has a Hamiltonian-cycle.
    \item If the answer is yes, then we don't need that edge, so we forget about it. :)
    \item If the answer is no, then we put the edge back and never touch it again.
    \item Repeat this for all edges.
    \item The edges remaining after all of this will be exactly one of the Hamiltonian-cycles of the graph.
    \item We ran the $HAM$ solver as many times as there were edges, exactly once, so this is a polynomial time algorithm.
\end{itemize}

