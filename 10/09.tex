\subsection {Session 10, Exercise 9}

\label{session_10_exercise_9}

\lineparagraph {Exercise}

We want to find the longest common consecutive subsequence of two character sequences $a_1a_2 \dots{} a_n$ and
$b_1b_2 \dots{} b_m$. That is, we need to find pair of indices $1 \leq{} i \leq{} n$ and $1 \leq{} j \leq{} m$ such that $a_{i+1} = b_{j+1}, a_{i+2} = b_{j+2}, \dots{}, a_{i+t} = b_{j+t}$ holds for the largest possible $t$. Give an  $O(mn)$ running time algorithm solving this problem.

\lineparagraph {Solution}

\begin{itemize}
    \item Notice how this is the same problem as in \nameref{session_10_exercise_8}, the only difference is that we are looking for the index pair, for which the maximum length of common subsequence is achieved.
    \item So at first, we do the same algorithm and calculate the $A$ matrix for the input words $a$ and $b$.
    \item Then, we find the cell with the maximum value, for example $A[i,j] = k$. However, this $k$ means that starting from $a_i$ and $b_j$, \textbf{backwards} up to $k$ steps the strings are equal. And the exercise is asking for the starting indexes, not the ending. So the answer is not $i$ and $j$, but $i-(k-1)$ and $j-(k-1)$.
    \item As the saying goes, ''There are two hard things in computer science: cache invalidation, naming things, and off-by-one errors.'' To quickly chek that our indexing is correct, I like to do the following:
    \item The starting index is $i-(k-1)$, the end is $i$ and this is correct if there are $k$ characters between these two.
    \item I remember the following formula: Between numbers $a$ and $b$, (counting $a$ and $b$ too), there are exactly $b-a+1$ numbers.
    \item So between $i-(k-1)=i-k+1$ and $i$ (counting both) there are exactly $i-(i-k+1)+1=i-i+k-1+1=k$ numbers, so this is correct.
\end{itemize}