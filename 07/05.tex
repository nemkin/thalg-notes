\subsection {Session 7, Exercise 5}

\lineparagraph {Exercise}

Let $\Sigma = \{0, 1, +\}$. Sketch a Turing machine that on an input of form $x + y$ where $x$, $y \in \{0, 1\}^*$ are nonempty bitstrings stops in finite time and when it stops on its 5th tape the sum of binary numbers x and y stands. Give an upper bound on its running time.

\lineparagraph {Solution}

\begin{itemize}
    \item Mark the beginning of tapes 2,3,4 tapes with an $X$.
    \item We first copy the input up until the $+$ character to the second tape. If no $+$ is found the input is rejected.
    \item We then copy the second part of the input after the $+$ character until the first emtpy cell. If there is another $+$ found, we reject.
    \item Now step with heads $2$ and $3$ 1 step backwards, to stand on the least significant bit of both $x$ and $y$.
    \item We do the method of summing two numbers we learnt in primary school. We store the current carry bit in our current state: $C_0$ and $C_1$. There are $8$ ($12$) possibilities:
    \begin{itemize}
        \item If the current state is $C_0$:
        \begin{itemize}
            \item If both heads see a $0$: We write a $0$ on the 4th tape and stay in $C_0$.
            \item If one head sees a $0$ or an empty cell and the other a $1$: We write a $1$ on the 4th tape and stay in state $C_0$.
            \item If both heads see a $1$: We write a $0$ on the 4th tape and move to state $C_1$.
            \item If both heads see an empty cell: Computation is done here, move to the copying stage.
        \end{itemize}
        \item If the current state is $C_1$:
        \begin{itemize}
            \item If both heads see a $0$: We write a $1$ on the 4th tape and move to $C_0$.
            \item If one head sees a $0$ or an empty cell and the other a $1$: We write a $0$ on the 4th tape and stay in state $C_1$.
            \item If both heads see a $1$: We write a $1$ on the 4th tape and stay in state $C_1$.
            \item If both heads see an empty cell: Since we still have a carry bit, we write that down on the 4th tape, then computation is done here, move to the copying stage.
        \end{itemize}
    \end{itemize}
    \item And if the current head saw a $0$ or a $1$ we move it to the left for both tape $2$ and $3$, while the head on tape $4$ moves to the right.
    \item After finishing with the computation we will have the required sum on the 4th tape, however the least significant bit will be on the first place. We need to reverse it.
    \item We can done this by copying from the current (last) position on the 4th tape to the 5th, by moving the 4th head to the left and the 5th head to the right, step-by-step.
\end{itemize}

This computation is done in $O(n)$, since copying to tape 2 and 3 is done in $O(n)$, then summing is done in $O(n)$ and finally reversal is also done in $O(n)$. (The resulting sum's length will not exceed the sum of the input $x$ and $y$ number's length in binary form.)