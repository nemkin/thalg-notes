\subsection {Session 7, Exercise 3}

\lineparagraph {Exercise}

Sketch a $1$-tape Turing machine for the language $\{0^{2^n} | n \geq{} 0\}$.

\lineparagraph {Solution}

We have to implement a TM that can divide by $2$ on a single tape.

We will be moving back and forth on the single tape, so we need to make sure we do not fall off
at the beginning of the tape. We can achieve this by two ways:

\begin{itemize}
    \item By replacing the first $0$ with a special character, for example $Z$, which indicates both that that is a $Z$, but also indicates the beginning of the tape. The empty string is not accepted, since $0^{2^0} = 0^1$, so that can immediately go to a special trap state to be rejected.
    \item By just replacing the first zero with an $X$ and moving the head to the first empty cell, writing a $0$ it, then moving back on the $X$. Handle the empty string similarly as previously.
\end{itemize}

I will be using the first method, since it will allow for an easier success check later.

We then move the head to the left and at every second step we replace the $0$ with a character $R$, to mark it as removed. This is done by cycling between two states, the first transition just stepping over a $0$, the second replacing it with the $R$. If there is no $0$ to be replaced, the division fails (the current number of $0$'s is odd), which means that we should reject).

When we reach the empty cell we move our head left until we reach $Z$.

In the upcoming runs we have to do the same thing as previously, however now we need to ignore $R$'s already placed, which can be done by adding two loops on the two states mentioned that just skip over any $R$ characters on the tapes.

Every single run of our subprocess results in dividing the number of $0$'s by two, which if the number was a power of $2$ should end up with a single zero: the specially marked $Z$ character at the beginning of the tape.

When we move to the right from the $Z$ cell and we find an empty cell, that means that we are done and can accept the input.
